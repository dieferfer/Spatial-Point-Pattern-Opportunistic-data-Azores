
\documentclass[12pt]{article}
\usepackage[a4paper, margin=2.5cm]{geometry}
\usepackage{hyperref}
\usepackage{xcolor}
\usepackage{fontspec}
\usepackage{titlesec}
\setmainfont{Latin Modern Roman}

\titleformat{\section}{\large\bfseries}{\thesection}{1em}{}
\titleformat{\subsection}{\normalsize\bfseries}{\thesubsection}{1em}{}

\title{\textbf{README -- Spatial Point Pattern Models for Opportunistic Cetacean Data in the Azores}}
\date{}

\begin{document}

\maketitle

\section*{Project Overview}
This repository contains the R code and resources used in the Master's thesis:

\textbf{``Using Bayesian Spatial Point Processes to Model Cetacean Distribution from Presence-Only Data''} \\
\textit{Master’s Program in Biostatistics – Universitat de València (2024/2025)} \\
Author: \textbf{Diego Fernández Fernández}

\bigskip

The project explores and models the spatial distribution of cetaceans using Bayesian Log-Gaussian Cox Processes (LGCP) based on opportunistic presence-only data. The focus is on the Azores archipelago.

\section*{Repository Structure}
\begin{itemize}
  \item \texttt{01\_define\_area.R}: Define and simplify the study area
  \item \texttt{02\_download\_occurrences.R}: Download and clean OBIS / GBIF data
  \item \texttt{03\_env\_variables.R}: Download and process environmental .nc data
  \item \texttt{04\_extract\_covariates.R}: Link observations with environmental variables
  \item \texttt{05\_model\_LGCP.R}: Fit LGCP models with INLA/inlabru
  \item \texttt{06\_predict\_intensity.R}: Generate predictive maps
  \item \texttt{/outputs/}: Maps and model summaries
  \item \texttt{/data/}: Raw and processed datasets
\end{itemize}

\section*{Methods Summary}
\begin{itemize}
  \item \textbf{Model}: Log-Gaussian Cox Process (LGCP)
  \item \textbf{Inference}: Integrated Nested Laplace Approximation (INLA)
  \item \textbf{Software}: \texttt{R}, \texttt{inlabru}
  \item \textbf{Covariates}: Bathymetry, SST, Salinity, Mixed Layer Depth, Zonal Currents
  \item \textbf{Data Source}: \href{https://marine.copernicus.eu/}{Copernicus Marine Service}
  \item \textbf{Species}: \textit{Delphinus delphis}, \textit{Physeter macrocephalus}
\end{itemize}

\section*{Getting Started}
Follow the scripts in numeric order from \texttt{01} to \texttt{06}. You will need:

\subsection*{Requirements}
\begin{itemize}
  \item \texttt{R >= 4.3}
  \item Required packages: \texttt{sf, terra, tidyverse, ncdf4, stars, lubridate, viridis}
  \item For INLA and inlabru:
\begin{verbatim}
install.packages("INLA", repos = c(getOption("repos"),
  INLA = "https://inla.r-inla-download.org/R/stable"))
install.packages("inlabru")
\end{verbatim}
\end{itemize}

\section*{License}
This work is licensed under the Creative Commons Attribution-NonCommercial 4.0 International License (CC BY-NC 4.0).

\section*{Acknowledgements}
\begin{itemize}
  \item Universitat de València – Facultad de Ciencias Matemáticas
  \item Supervisors: David V. Conesa Guillén, Antonio López Quílez
  \item Project REDUCE (IEO-CSIC)
\end{itemize}

\section*{Contact}
For questions or collaboration ideas, reach out via GitHub or LinkedIn.

\bigskip

\textit{“Modeling where science meets the sea.”} 🌊

\end{document}
